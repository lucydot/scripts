% * * * * * * * * * * * * * * * * * * * * * * * * * * * * * * * * * * * 
% *
% * Documentation classe VaspRun
% *
% * Germain Vallverdu
% * Mer 6 Oct 2010
% * germain.vallverdu@univ-pau.fr
% *
% * * * * * * * * * * * * * * * * * * * * * * * * * * * * * * * * * * *

\documentclass[10pt,a4paper,fleqn]{article}

% * * * * * * * * * * * * * * * * * * * * * * * * * * * * * * * * * * * 
% *
% * preambule
% *
% * * * * * * * * * * * * * * * * * * * * * * * * * * * * * * * * * * *

% * * * * * * * * * * * * * * * * * * * * * * * * * * * * * * * * * * * * * * *
% * 
% * POLICE, ENCODAGE, LANGUE, FIGURES, COULEURS
% * 
% * * * * * * * * * * * * * * * * * * * * * * * * * * * * * * * * * * * * * * *

\usepackage[utf8]{inputenc}
\usepackage[T1]{fontenc}
\usepackage[francais]{babel}

% police
\renewcommand{\familydefault}{\sfdefault}
\usepackage{newcent}

\usepackage[svgnames]{xcolor}
\usepackage{graphicx,tikz}

\usepackage{mes_macros}

% * * * * * * * * * * * * * * * * * * * * * * * * * * * * * * * * * * * * * * *
% * 
% * FORMAT DES PARAGRAPHE, ELARGISSEMENT DES TABLEAUX
% * 
% * * * * * * * * * * * * * * * * * * * * * * * * * * * * * * * * * * * * * * *

\setlength{\parskip}{1.5ex plus 0.5ex minus 0.5ex}	% espacement entre les paragraphes
\setlength{\parindent}{0em}        			% pas d'indentation
\usepackage{setspace}              			% pour gérer l'interligne

\usepackage{tabularx}

\renewcommand{\arraystretch}{1.4}  			% élargir les tableaux

% * * * * * * * * * * * * * * * * * * * * * * * * * * * * * * * * * * * * * * *
% * 
% * HYPERREF
% * 
% * * * * * * * * * * * * * * * * * * * * * * * * * * * * * * * * * * * * * * *

\usepackage{hyperref}

\hypersetup{% 
pdftex,%			Sets up hyperref for use with the pdftex program
colorlinks=false,% 		active les liens
citecolor=RoyalBlue,
linkcolor=RoyalBlue,
pdfborder={0 0 0},%		bordure des liens
pdfauthor={Germain Vallverdu},%
pdftitle={Classe python VaspRun pour le post-traitement de calculs VASP},% 
pdfkeywords={vasp,python,post-traitement,dos,bandes},%
pdfcreator={PDFLaTeX},%
pdfproducer={PDFLaTeX},%
}

% * * * * * * * * * * * * * * * * * * * * * * * * * * * * * * * * * * * 
% *
% * format de la page
% *
% * * * * * * * * * * * * * * * * * * * * * * * * * * * * * * * * * * *

\usepackage[top=2.5cm,bottom=2.5cm,left=2.5cm,right=2.5cm]{geometry}

\usepackage{fancyhdr}
\pagestyle{fancy}
\fancyhead{}                         % supprime toutes les entetes
\fancyfoot{}                         % supprime tous les pieds de page
\fancyhead[L]{\sffamily\itshape VaspRun : post-traitement VASP}
\fancyfoot[LE,RO]{\footnotesize\thepage}          % le numero de page en bas a droite ou gauche
\renewcommand{\headrulewidth}{0.4pt}
\renewcommand{\footrulewidth}{0pt}

% redefini le syle plain
\fancypagestyle{plain}{%
\fancyhf{}                           % clear all header and footer fields
\fancyfoot[RE,RO]{\thepage}
\renewcommand{\headrulewidth}{0pt}
\renewcommand{\footrulewidth}{0pt}}

% * * * * * * * * * * * * * * * * * * * * * * * * * * * * * * * * * * * 
% *
% * package listing
% *
% * * * * * * * * * * * * * * * * * * * * * * * * * * * * * * * * * * *

\usepackage{listings}
\lstset{% general command to set parameter(s)
language=,                                     % langage
identifierstyle=,                               % rien
basicstyle=\color{black!5}\ttfamily,              % style du texte
stringstyle=,             % styme typewriter pour string
keywordstyle=,        % style des mots clef
commentstyle=,                      % style commentaire
backgroundcolor=,                  % couleur arrière plan
}

% * * * * * * * * * * * * * * * * * * * * * * * * * * * * * * * * * * *

% * * * * * * * * * * * * * * * * * * * * * * * * * * * * * * * * * * * 
% *
% * package listing
% *
% * * * * * * * * * * * * * * * * * * * * * * * * * * * * * * * * * * *

\usepackage{textcomp}
\usetikzlibrary{calc}
\usepackage{verbatim}

\makeatletter
\newwrite{\temp@listing@out}
\newenvironment{code}{%
\immediate\openout\temp@listing@out \jobname-temp.tex
\bgroup\@bsphack
\let\do\@makeother\dospecials
\catcode`\^^M\active
\def\verbatim@processline{%
\immediate\write\temp@listing@out{\the\verbatim@line}%
}%
\verbatim@start}
{\@esphack\egroup
\immediate\closeout \temp@listing@out
\begin{tikzpicture}
    \node [fill=gray!10,rectangle,inner xsep=10pt,inner ysep=10pt] (box)
       {\begin{minipage}{\dimexpr\textwidth-21.66pt\relax}
       \lstinputlisting{\jobname-temp.tex}
       \end{minipage}
       };
    \node[text=white,fill=gray,rectangle, shading=ball, ball color=gray, above right] (title) at ($(box.north west)+(-0.03,0)$){\textbf{CODE}};
    \draw[color=gray!50!black,very thick] (box.north west)--(box.south west)--(box.south east);
\end{tikzpicture}

}

\newenvironment{console}{%
\immediate\openout\temp@listing@out \jobname-temp.tex
\bgroup\@bsphack
\let\do\@makeother\dospecials
\catcode`\^^M\active
\def\verbatim@processline{%
\immediate\write\temp@listing@out{\the\verbatim@line}%
}%
\verbatim@start}
{\@esphack\egroup
\immediate\closeout \temp@listing@out
\vspace*{-0.6\baselineskip}
\begin{center}
\begin{tikzpicture}
	\node[rectangle,rounded corners,fill=black!90, inner xsep=2ex,
	text width=0.9\textwidth,text=black!5] (box)
       {
       \lstinputlisting{\jobname-temp.tex}
       };
\end{tikzpicture}
\end{center}

}

\makeatother

%\usepackage{fancyvrb}
%
%\newenvironment{console}
%	{\SaveVerbatim{VerbEnv}}
%	{\endSaveVerbatim\parindent0pt
%
%	\begin{center}
%		\begin{tikzpicture}
%			\node[rectangle,rounded corners,fill=black!20,inner sep=4mm,
%			text width=0.9\textwidth,text=black!5] (box)
%			{
%			\UseVerbatim{VerbEnv}
%			};
%			\node[text=black!90,rectangle,rounded corners,fill=white,draw=black!90,very thick] 
%			at (box.169) {\textbf{CONSOLE}} ;
%		\end{tikzpicture}
%	\end{center}
%}


\usetikzlibrary{shapes}

\graphicspath{{./images/}}

\title{Classe python VaspRun pour le post-traitement de calculs VASP}
\author{Germain Vallverdu}
\date{\today}

% * * * * * * * * * * * * * * * * * * * * * * * * * * * * * * * * * * * 
% *
% * debut document
% *
% * * * * * * * * * * * * * * * * * * * * * * * * * * * * * * * * * * *

\begin{document}
\renewcommand{\labelitemi}{\hspace{3ex} $\bullet$} 

\maketitle

La classe \verb!VaspRun! est une classe python permettant de faire du post traitement de calculs
VASP . Elle extrait les informations du fichier \verb!vasprun.xml! créé par VASP à la fin du calcul.
La raison du choix de lire les données sur le fichier \verb!vasprun.xml! plutôt que sur les différents
fichiers de sortie de VASP est que le fichier xml est un ensemble cohérent. Il contient à la fois
les paramètres du calculs et les résultats. En utilisant ce fichier on est donc certain de la
correspondance entre les résultats et les paramètres du calcul.

Il est inutile de connaître le langage python pour l'utiliser. La classe \verb!VaspRun! peut être vue
comme une simple boite à outils. Ce document décrit les possibilités offertes par cet outil.

\tableofcontents

\newpage

\section{Utilisation}

\subsection{Généralités :}

Pour l'utiliser on procède en quatre étapes dans un terminal :

\begin{enumerate}
	\item On lance python
	\item On charge le fichier VASP.py qui contient les outils de post-traitement.
	\item On crée un nouveau calcul VASP (pour parler technique : on crée une
		instance de la classe \verb!VaspRun!).
	\item On utilise un des outils (on appelle une méthode de la classe \verb!VaspRun!).
\end{enumerate}

\subsection{Un exemple détaillé :}

Un petit exemple, pour extraire les bandes d'énergie :

\begin{enumerate}
	\item On lance python

\begin{console}
[gvallver@kiwi] > python3
\end{console}

	\item On charge le fichier VASP.py

\begin{console}
>>> import VASP
\end{console}

	\item On crée un calcul que l'on appelle \verb!calcul!, le nom est sans importance. C'est
		ici que l'on appelle la classe \verb!VaspRun!, comme elle est contenu dans le
		fichier VASP.py on l'appelle en indiquant : \verb!VASP.VaspRun()!

\begin{console}
>>> calcul = VASP.VaspRun()
\end{console}

	\item on demande d'extraire les bandes d'énergie :

\begin{console}
>>> calcul.extraireBandes()
\end{console} 

\end{enumerate}

Les étapes de 1 à 3 ne sont réalisées qu'une seule fois au départ. On va ensuite utiliser les outils
que l'on souhaite de la même manière qu'à l'étape 4. On utilise la syntaxe :

\begin{console}
	mon_calcul . un_outil( options )
\end{console}

\subsection{L'exemple complet, une vrai exécution :}

Voilà ce qui se passe lorsqu'on entre la série de commande précédente. On peut voir que lors de la
création du calcul (lorsqu'on instancie la classe \verb!VaspRun!) des informations concernant le
calcul s'affichent. On a entre autre le contenu du fichier \verb!INCAR! puis le nombre et le type
d'atome.

\begin{console}
[gvallver@kiwi]> python3
Python 3.1.2 (r312:79147, Apr 15 2010, 12:35:07) 
[GCC 4.4.3] on linux2
Type "help", "copyright", "credits" or "license" for more information.
>>> import VASP
>>> calcul = VASP.VaspRun()

# Fichier xml du calcul : vasprun.xml
        * lecture du fichier xml
        * Lecture des mots clefs du calcul
        * Lecture des points K du calcul
        * Lecture des donnees sur les atomes du calcul

# fichier INCAR du calcul
     ENCUT = 400.0         ISTART = 0               NELM = 200       
      PREC = normal          ISIF = 3             IBRION = 2         
    EDIFFG = -0.01         ISMEAR = 0              LREAL =  auto     
     EDIFF = 1e-06            NSW = 100            SIGMA = 0.05      

# caracteristiques du systeme :
        * nombre d'atomes  : 8
        * nombre de types  : 1
        * liste des atomes : Si, Si, Si, Si, Si, Si, Si, Si

# Liste des types d'atomes du calcul et de leur caracteristiques :
Atome Si : 
        masse                    : 28.085
        Nbre electron de valence : 4.0
        type                     : 1
        Pseudo potentiel         :  PAW_PBE Si 05Jan2001                   

>>> run.extraireBandes()
# Lecture des bandes d'energies
        * ISPIN            = 1
        * nombre de bandes = 8

# extraction des bandes d'energie
Fichier(s) cree(s) : 1
        bandes.dat

>>>
\end{console}

\subsection{Autres syntaxes :}

Il existe une autre syntaxe possible pour créer un calcul qui peut sembler plus simple.
L'utilisation de l'une ou l'autre n'a pas d'importance.

\begin{console}
[gvallver@kiwi]> python3
Python 3.1.2 (r312:79147, Apr 15 2010, 12:35:07) 
[GCC 4.4.3] on linux2
Type "help", "copyright", "credits" or "license" for more information.
>>> from VASP import *
>>> calcul = VaspRun()
...
\end{console}

Il est possible que le fichier xml du calcul ne s'appelle pas \verb!vasprun.xml!, par exemple
parce qu'il a été renommé. Dans ce cas, lors de la création du calcul, il suffit de préciser le nom
(ou le chemin) du fichier xml entre guillemets dans les parenthèse. Voici un exemple où le fichier
s'appelle \verb!bandes.xml!.

\begin{console}
[gvallver@kiwi]> python3
Python 3.1.2 (r312:79147, Apr 15 2010, 12:35:07) 
[GCC 4.4.3] on linux2
Type "help", "copyright", "credits" or "license" for more information.
>>> from VASP import *
>>> calcul = VaspRun(''bandes.xml'')

# Fichier xml du calcul : bandes.xml
        * lecture du fichier xml
        * Lecture des mots clefs du calcul
        * Lecture des points K du calcul
        * Lecture des donnees sur les atomes du calcu
...
\end{console}

\subsection{Scripts :}

Si on doit faire un traitement répétitif il est possible de placer l'ensemble des commandes dans un
fichier à exécuter avec python. Reprenons l'exemple de l'affichage du fichier \verb!INCAR! du
calcul. On place dans un fichier \verb!script.py! les lignes suivantes (les lignes commençant par \#
sont des commentaires) :

\begin{lstlisting}[language=python,
	keywordstyle=\bfseries\color{FireBrick},
	basicstyle=\ttfamily,commentstyle=\color{Navy!60}]
# on charge VASP
from VASP import *

# on cree le calcul
calcul = VaspRun(``bandes.xml'')

# on appelle la methode pour afficher le fichier INCAR
calcul.extraireBandes()
\end{lstlisting}

On exécute ensuite le script avec python dans un terminal :

\begin{console}
[gvallver@kiwi]> python3 script.py

# Fichier xml du calcul : bandes.xml
        * lecture du fichier xml
        * Lecture des mots clefs du calcul
        * Lecture des points K du calcul
        * Lecture des donnees sur les atomes du calcul

# fichier INCAR du calcul
     ENCUT = 450.0         ISTART = 1             SYSTEM =  "Si bulk"
      PREC = medium        ICHARG = 11              ALGO =  FAST     
    ISMEAR = 0              EDIFF = 1e-06          SIGMA = 0.05      

# caracteristiques du systeme :
        * nombre d'atomes  : 2
        * nombre de types  : 1
        * liste des atomes : Si, Si

# Liste des types d'atomes du calcul et de leur caracteristiques :
Atome Si : 
        masse                    : 28.085
        Nbre electron de valence : 4.0
        type                     : 1
        Pseudo potentiel         :  PAW_PBE Si 05Jan2001                   


# Lecture des bandes d'energies
        * ISPIN            = 1
        * nombre de bandes = 8

# extraction des bandes d'energie
Fichier(s) cree(s) : 1
        bandes.dat

[gvallver@kiwi]> 	   
\end{console}

\section{Liste des outils}

En python les ``outils'' disponnibles dans une classe sont appelé des méthodes. Voici donc la liste
des méthodes, ou outils, disponibles dans la classe \verb!VaspRun! et leur utilisation. Les noms des
méthodes sont assez explicites. On donne d'abord la liste des méthodes, leur utilisation est
décrites plus loin.

On rappelle que pour utiliser un outil quel qu'il soit, on utilise la syntaxe :

\begin{console}
	mon_calcul . un_outil( options )
\end{console}

\subsection{Liste des méthodes :}

\begin{itemize}
	\item Obtenir des informations sur les paramètres du calculs :
		\begin{itemize}
			\item getINCAR()
			\item getKPOINTS()
			\item listerMotsClefs()
			\item valeurMotClef( ``Mot Clef'' )
			\item listerGroupesMotsClefs()
			\item groupeMotsClefs( ``titre du groupe'')
			\item getInfoAtomes() \\
		\end{itemize}

	\item Obtenir les bandes d’énergies :
		\begin{itemize}
			\item extraireBandes( parDirection= True/False )
			\item lectureBandes() \\
		\end{itemize}

	\item Obtenir les densités d’états totale et projetées :
\end{itemize}

\subsection{Obtenir de l'aide :}

Pour obtenir de l'aide directement dans la consôle python (après avoir lancé python) on a deux
possibilités :

\begin{itemize}
	\item On utilise la fonction \verb!help()! de python
	\item On appelle la méthode \verb!help()! de la classe \verb!VaspRun!.
\end{itemize}

Les deux possibilités rapporte les mêmes informations. La fonction \verb!help()! de python contient
cependant plus d'informations techniques alors que celle de la classe \verb!VaspRun! contient
uniquement le nom des méthodes, leurs but et leur syntaxe.

Voici les syntaxes possibles :

\begin{console}
[gvallver@kiwi]> python3
Python 3.1.2 (r312:79147, Apr 15 2010, 12:35:07) 
[GCC 4.4.3] on linux2
Type "help", "copyright", "credits" or "license" for more information.
>>> from VASP import *
>>> VaspRun.help()
. . . affiche l'aide interne de VaspRun
>>>
>>> help(VaspRun)
. . . affiche l'aide python sur VaspRun
>>>
>>> import VASP
>>> VASP.VaspRun.help()
. . . affiche l'aide interne de VaspRun
>>>
>>> help(VASP.VaspRun)
. . . affiche l'aide python sur VaspRun
>>>
>>> calcul = VASP.VaspRun()
>>> help(calcul)
. . . affiche l'aide python sur VaspRun
>>>
\end{console}

\subsection{Obtenir des informations sur les paramètres du calculs :}

Ces méthodes permettent d'obtenir des informations sur les paramètres du calcul et le système
étudié.

\begin{itemize}
	\item getINCAR() \\

		Cette méthode affiche l'ensemble des mots clefs présents dans le fichier INCAR
		du calcul. \\

	\item getKPOINTS() \\

		Affiche les informations du fichiers KPOINTS \\

	\item listerMotsClefs() \\

		Cette méthode interroge la base de données contenant l'ensemble des valeurs des
		mots clefs de VASP utilisés dans le calcul et affiche la liste des mots clefs.
		(voir\par \verb!valeurMotClef( "Mot Clef" )! ). \\

	\item valeurMotClef( "Mot Clef" ) \\

		Cette méthode interroge la base de données contenant l'ensemble des valeurs des
		mots clefs de VASP utilisés dans le calcul et affiche la valeur du mot clef
		demandé (voir\par \verb!listerMotsClefs()! ). \\

	\item groupeMotsClefs("titre du groupe") \\

		Cette méthode interroge la base de données contenant les mots clefs du calcul
		et affiche la liste des mots clefs contenu dans le groupe dont le titre est donné en
		argument (voir \verb!listerGroupesMotsClefs()! ). \\

	\item listerGroupesMotsClefs() \\

		Cette méthode interroge la base de données contenant les mots clefs du calcul
		et affiche la liste des titres des groupes de mots clefs (voir \par
		\verb!groupeMotsClefs( "titre" )! ) \\

	\item getInfoAtomes() \\

		Affiche des informations sur le système : nombre d'atomes, types, pseudo \ldots \\

\end{itemize}

\subsection{Obtenir les bandes d'énergies :}

Ces méthodes permettent d'extraire les bandes d'énergie. Il n'est pas nécessaire d'appeler la
méthode de lecture des bandes d'énergie avant celle pour extraire les bandes d'énergie.

\begin{itemize}
	\item extraireBandes( parDirection= True/False ) \\
		
		Méthode permettant d'extraire les bandes d'énergie du fichier xml et de les
		écrire dans un format pratique pour le tracer. \\

		\begin{itemize}
			\item \verb!calcul.extraireBandes()! ou 
				\verb!calcul.extraireBandes(parDirection=False )!
				\\

				extrait toutes les bandes dans le fichier "bandes.dat". Sur le
				fichier "bandes.dat", la colonne 1 donne l'indice du point k entre 0
				et 1 (indice du points k / nbre de points k). Les colonnes de 2 à
				NBANDS (nombre de bandes) contiennent les valeurs des bandes
				d'énergie pour ce point k en eV.

				Si ISPIN = 2, un fichier est créé pour chaque spin. \\

			\item \verb!calcul.extraireBandes( parDirection=True )!  \\

				extrait toutes les bandes par direction dans l'espace réciproque et
				crée un fichier par direction. Dans chaque fichier les 3 premières
				colonnes donnent les coordonnées du points k. Les colonnes de 4 à
				NBANDS (nombre de bandes) contiennent les valeurs des bandes
				d'énergie pour ce point k en eV. \\
		\end{itemize}

	\item lectureBandes() \\

		Méthode permettant de lire les bandes d'énergie sur le fichier xml du calcul.
		Après exécution de cette méthode, on dispose des bandes d'énergie dans la liste
		Bandes sous la forme :
    
\begin{console}
from VASP import *
calcul = VaspRun()
calcul.bandes[ spin ][ point k ][ bandes ][ i ]
\end{console}

\begin{tabularx}{0.95\textwidth}{lcX}
    
                    spin  &  : & 0 ou 1, seule la valeur 0 est disponnible pour les calculs
                            avec spin non polarisé (ISPIN = 1). Pour les calculs spin
                            polarisé (ISPIN = 2), spin = 0 donne les bandes pour les
                            spin alpha et spin = 1 pour les spin beta. \\
    
                    point k & : & 0 -> nombre de point K\par
                            le nombre de point k est calcul.nbrePointsK
                            calcul.nbreDirectionsPointsK : nombre de directions de points k
                            calcul.Ndivision             : nombre de points k par direction
                            Les points k sont listés direction après direction en
                            donnant l'ensemble des points k sur chaque direction. \\
    
                    bandes & : & 0 -> NBANDS-1 \\
    
                    i   &    : & 0 ou 1 \par
                            i = 0 -> valeurs de l'énergie
                            i = 1 -> occupation de la bande \\
\end{tabularx}

\end{itemize}

\subsection{Obtenir les densités d'états totale et partielles :}

\begin{itemize}
	\item extraireDOS( partielles = True/False ) \\

		Méthode permettant d'extraire la densité d'états totale et les densités d'états
		partielles projetées sur les atomes et de les écrire dans un format pratique pour
		les tracer.  \\

		\begin{itemize}
			\item \verb!calcul.extraireDOS()! ou 
				\verb!calcul.extraireDOS( partielles=False )! \\
				
				extrait la DOS totale dans le fichier "dosTotale.dat". Sur le
				fichier "dosTotale.dat", la colonne 1 donne l'énergie, la colonne 2
				donne la densité d'états et la colonne 3 donne l'intégrale de la
				densité d'états. \par
				Si ISPIN = 2, un fichier est créé pour chaque spin. \\

			\item \verb!calcul.extraireDOS( partielles = True ) ! \\

				extrait la DOS totale dans le fichier "dosTotale.dat" et les DOS
				partielles sur des fichiers séparés. Pour les DOS partielles, un
				fichier par atome est créé portant le nom de l'atome et son numéro
				correspondant à l'ordre dans lequel ils sont donnés dans le POSCAR.
				Les fichiers contenant les DOS partielles ont en colonne 1 l'énergie
				et en colonne 2 -> 9 les DOS partielles projetées sur les OA dans
				l'ordre : s py pz px dxy dyz dz2 dxz dx2. \par
				Si ISPIN = 2, un fichier est créé pour chaque spin. \\
		\end{itemize}



	\item lectureDOS() \\

		Lecture de la densité d'états totale et des densités d'états partielles sur le
		fichier xml du calcul. Après exécution de cette méthode, on dispose de la densité
		d'états totale dans la liste dosTotale et des densités d'états partielles dans la
		liste dosPartielles. Elles sont de la forme :


\begin{console}
from VASP import VaspRun
calcul = VaspRun()
calcul.dosTotale[ spin ][ E ][ i ]
\end{console}
    
\begin{tabularx}{0.95\textwidth}{lcX}
                    spin   & : & 0 ou 1, seule la valeur 0 est disponnible pour les calculs
                            avec spin non polarisé (ISPIN = 1). Pour les calculs spin
                            polarisé (ISPIN = 2), spin = 0 donne les bandes pour les
                            spin alpha et spin = 1 pour les spin beta. \\
    
                    E      & : & indice de parcours de la DOS pour un spin (valeurs de l'énergie) \\
    
                    i      & : & 0 -> 2\par
                            i = 0 -> valeurs de l'énergie\par
                            i = 1 -> densité d'état\par
                            i = 2 -> intégrale de la DOS \\
\end{tabularx}

\vspace{5mm}

\begin{console}
calcul.dosPartielles[iat][ spin ][ E ][ i ]
\end{console}
    
\begin{tabularx}{0.95\textwidth}{lcX}
		    iat    & : & numéro de l'atome \\
                    spin   & : & 0 ou 1, seule la valeur 0 est disponnible pour les calculs
                            avec spin non polarisé (ISPIN = 1). Pour les calculs spin
                            polarisé (ISPIN = 2), spin = 0 donne les bandes pour les
                            spin alpha et spin = 1 pour les spin beta. \\
    
                    e      & : & indice de parcours de la DOS pour un spin (valeurs de l'énergie) \\
    
                    i      & : & 0 -> 8 valeurs de la DOS sur chaque OA \par
		    0 = $s$, 1 = $p_y$, 2 = $p_z$, 3 = $p_x$, 4 = $d_{xy}$, 5 = $d_{yz}$, 6 = $d_{z^2}$,
		    7 = $d_{xz}$, 8 = $d_{x^2-y^2}$ \\
\end{tabularx}

\end{itemize}

\newpage
\section{Liste des attributs}

Cette section liste les attributs des objets de type \verb!VaspRun!.

\begin{tabularx}{\textwidth}{|l|>{\itshape}c|X|}
	\hline
	\textbf{Attribut} & \textbf{type} & \centering \textbf{Définition} \tabularnewline
	\hline
	self.fichierXML & string & nom du fichier xml \\
	self.xml & dom & objet xml créé à partir du fichier xml \\
	\hline
	self.erreur & bool & vrai si le fichier xml est introuvable \\
	self.DOSlue &bool &  contrôle si la DOS totale a été lue \\
	self.DOSPartiellesLues & bool & contrôle si les DOS partielles ont été lues \\
	self.BandesLues & bool & contrôle si les bandes d'énergie ont été lues \\
	\hline
	self.allMotsClefs & dic & dictionnaire des mots clefs du calcul \\
	self.INCAR & dic & dictionnaire des mots clefs du fichier INCAR \\
	self.groupesMotsClefs & dic & dictionnaire des groupes de mots clefs, chaque groupe est repéré par
	le titre du groupe.\\
	\hline
	self.Natomes & int & Nombre d'atomes dans le calcul \\
	self.Ntypes & int & Nombre de type d'atomes dans le calcul \\
	self.atomesDuCalcul & UnAtome & Liste des atomes du calcul. Chaque élément de la liste est de type
	UnAtome et possède les attributs : nom, atomType, nElecValence, masse et pseudo. \\
	\hline
	self.eFermi & float & Energie en eV au niveau de Fermi \\
	self.dosTotale & list & Tableau contenant la DOS totale : \par
		\verb!dosTotale[spin][E][i]! \\
	self.dosPartielles & list & Tableau contenant les DOS partielles :\par 
	\verb!dosPartielles[iat][spin][E][i]! \\
	\hline
	self.bandes & list & Tableau contenant les bandes d'énergie :\par 
	\verb!Bandes[spin][point k][bandes][i]! \\
	\hline
	self.typePointsK & string & type de points k : Gamma (grille régulière) ou listgenerated (selon des
	lignes de symétrie). \\
	self.Ndivision & int & Nombre de points k le long de chaque ligne (listgenerated) ou le long de
	chaque direction (Gamma) \\
	self.nbrePointsK & int & Nombre totale de points k \\
	self.nbreDirectionsPointsK & int & Nombre de ligne de points k dans le cas listgenerated. \\
	self.PointsK & list & Dans le cas où self.typePointsK = listgenerated, self.PointsK est la liste
	des points k entre lesquels sont construites les lignes de points k. \\
	self.listePointsK & list & Liste de longueur self.nbrePointsK contenant les coordonnées des points
	k \\
	\hline
\end{tabularx}

\end{document}
